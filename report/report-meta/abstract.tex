\likeheader{реферат}
\pagestyle{plain}
Дипломная работа, 30 стр., 9 рис., 6 источников, 1 приложение

\textbf{Ключевые слова}: ДИФФЕРЕНЦИАЛЬНОЕ УРАВНЕНИЕ, ВАРИАЦИОННАЯ ПОСТАНОВКА, 
КОНЕЧНЫЕ ЭЛЕМЕНТЫ, FENICS, \\ PYTHON.

\textbf{Объект исследования} --- задача сопряжения уравнений гиперболического и параболического типов.

\textbf{Цель работы} --- численное решение задачи сопряжения уравнений гиперболического и параболического типов

\textbf{Методы исследования} --- метод конечных элементов.

\textbf{Результатами} являются: вычислительный алгоритм и программа решениия задачи сопряжения
уравнений гиперболического и параболического типов

\textbf{Область применения} --- приближенное решение дифференциальных \\ уравнений, математическое
моделирование процессов, протекающих в разнородных средах.

\likeheader{рэферат}

Дыпломная работа, 30 с., 9 мал., 6 крынiц, 1 дадатак.

\textbf{Ключавыя словы} ДЫФЕРЕНЦЫАЛЬНАЕ РАУНАННЕ, ВАРЫЯЦЫЕННАЯ ПАСТАНОУКА, КАНЧАТКОВЫЯ ЭЛЕМЕНТЫ,
FENICS, \\ PYTHON.

\textbf{Аб'ект даследавання} --- задача аб спалучэннi гiпербалiчнага i парабалiчнага раунання.

\textbf{Мэта работы} --- вылiковае рашэнне задачы аб спалучэннi гiпербалiчнага i парабалiчнага раунання.

\textbf{Метады даследавання} --- метад канчатковых элементау.

\textbf{Вынiкамi} з'яуляюцца: вылiковы алгарытм i праграма рашэння задачы аб спалучэннi гiпербалiчнага i парабалiчнага раунання.

\textbf{Вобласць прымяненя} --- прыблiзнае рашэнне дыференцыяльных раунанняу, матэматычнае
мадэляванне працэсау якiя праходзяць у разнастайных ассяроддзях. \\

\newpage

\likeheader{summary}

Graduate work. 30 p., 9 pic., 6 sources, 1 appendix

\textbf{Key words}: DIFFERENCIAL EQUATION, VARIATION FORMULATION,\\ FINITE ELEMENTS, FENICS, PYTHON.

\textbf{Research object}: problem about hyperbolic and parabolic equation \\conjugation.

\textbf{Work goal}: numerical solution of hyperbolic and parabolic equation \\ conjugation problem.

\textbf{Research methods} --- finite element method.

\textbf{Results} --- computational algorithm and program for solving hyperbolic
and parabolic equations conjugation problem.

\textbf{Use area} --- approximate solutions of differential equations, mathematical modeing of processing in various media.
