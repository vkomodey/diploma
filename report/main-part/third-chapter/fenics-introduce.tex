\subsection{Описание пакета FEniCS}

Для применения метода конечных элементов будем использовать пакеты FEniCS и FEniCSTools.
Пакет FEniCS состоит из большого количества библиотек(написанных на C++, которые транспилируются в python модули, что говорит о довольно быстрой реализации данного пакета) призванных упростить решения различных дифференциальных уравнений. 
Использование FEniCS подразумевает собой, что пользователь должен иметь абстрактные знания о методе конечных элементов(основной метод, которым FEniCS решает уравнения). Но на самом деле, FEniCS скрывает от пользователя конечную реализацию алгоритма. Таким образом, код приложения весьма лаконичен и понятен человеку, который знает лишь базовые понятия языка программирования python.
Несмотря на то, что все пакеты можно установить из репозиториев вашего дистрибутива(как то apt-get install fenics), я все же рекомендую делать это через исходный код самих пакетов. Таким образом мы можем подобрать корректную версию FEniCS и FEniCSTools.
Пакет FEniCS предназначен для решения задач методом конечных элементов в различных вариациях. 
FEniCS базируется на библиотеке Dolfin, которую так же нужно установить. FEniCS позволяет проводить сложные вычисления
вводя в программу лишь аналитический вид уравнения в аналитическом виде.

Так же FEniCS проводит дискретизацию области по пространству различными способами и с различными типами базисных функций(готовую сетку можно отдать FEniCS в входном файле, что дает довольно большой простор для действий). Пакет FEniCSTools используется для экстраполяции функции из подобласти $\Omega_1$ или $\Omega_2$ на всю подобласть $\Omega$.

Так же стоит отметить, что пакет FEniCS высокоуровневый и низкоуровневый одновременно, что позволяет настроить работу программы практически под любые действия, не используя при этом множество сторонних библиотек. FEniCS активно разрабатыватся в настоящее время, поэтому практически всегда можно найти решения возникающих проблем или задать вопрос разработчикам данного пакета.
