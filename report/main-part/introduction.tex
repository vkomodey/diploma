\newpage

\begin{center}
    \anonsection{ВВЕДЕНИЕ}
\end{center}

Необходимость рассмотрения сопряжения, когда на одной части области задано уравнение параболического типа, а на другой -- уравнение гиперболического типа, была впервые высказана И.М. Гельфандом в 1959 г[ИСТОЧНИК]. К задаче сопряжения приводит изучение электрических колебаний в проводах.

Такого рода задачи встречаются также при изучении движения жидкости в канале, окруженной пористой средой, в теории рапространения электромагнитных полей и в ряде других областей физики. Так, в канале гидродинамическое давление жидкости удовлетворяет волновому уравнению, а в пористой среде -- уравнению фильтрации, которое в данном случае совпадает с уравнением диффузии[ИСТОЧНИК]. При этом на границе канала выполняются некоторые условия сопряжения. Аналогичная ситуация имеет место для магнитной напряженности электромагнитного поля в указанной выше неоднородной среде[3]. Большой интерес представляет изучение влияния вязкоупругих свойств нефти на различные технологические процессы ее добычи. Если рассмотреть совместное движение различных несмешивающихся жидкостей в трещинах и пористых пластах с учетом вязкоупругих характеристик, то движение вязкоупругой и вязкой жидкостей в плоской горизонтальной трещине без учета поверхностных явлений описывается одномерным гиперболическим уравнением и уравнением теплопроводности с интегро-дифференциальными условиями на границе раздела движущихся жидкостей.

В монографии А.Г. Шашкова[ИСТОЧНИК] строится структуная модель теплопроводности в системе, составленной из теплоизолированных с боковой поверхности ограниченного и полуограниченного стержней, имеющих одинаковую температуру. На свободный конец системы поступает изменяющийся во времени тепловой поток. Температурное поле в ограниченном стержне описывается обычным уравнением теплопроводности, а в полуограниченном - гиперболическим уравнением. Теплофизические свойства стержней различны. В месте соприкосновения стержней имеет место идеальный тепловой поток. 

Большой интерес представляет изучение математических моделей, описывающих влияние растительного покрова на теплообменные процессы в почве и приземном воздухе, при котором возникает необходимость исследования задачи для двух уравнений: уравнения Аллера переноса влаги, предполагающего бесконечную скорость распространения возмущения, и уравнене Лыкова, учитывающего конечную его скорость.

За последние несколько десятилетий в математической литературе появилось значительное количество публикаций, посвященных задачам сопряжения по временной переменной. Достаточно полная библиография по этой теории содержится в монографиях Т.Д. Джураева[ИСТОЧНИК], М. Мамажанова [ИСТОЧНИК]. В приведенных выше работах задачи сопряжения двух уравнений по пространственной переменной в основном изучались для бесконечных или полубесконечных областей.

В настоящей работе решаются задачи о сопряжении гиперболического и параболического уравнений по пространственной переменной в конечных областях.

\newpage
