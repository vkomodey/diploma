\newpage

\begin{center}
    \anonsection{ВВЕДЕНИЕ}
\end{center}

Необходимость рассмотрения задач сопряжения, когда на одной части области задано уравнение параболического типа, а на другой -- уравнение гиперболического типа, была впервые высказана И.М. Гельфандом в 1959 г\cite{bib:gelfand}. Наприме, к задаче сопряжения приводит изучение электрических колебаний в проводах.

Такого рода задачи встречаются также при изучении движения жидкости в канале, окруженной пористой средой, в теории рапространения электромагнитных полей и в ряде других областей физики. Так, в канале гидродинамическое давление жидкости удовлетворяет волновому уравнению, а в пористой среде -- уравнению фильтрации, которое в данном случае совпадает с уравнением диффузии\cite{bib:leibezon}. При этом на границе канала выполняются некоторые условия сопряжения. Аналогичная ситуация имеет место для магнитной напряженности электромагнитного поля в указанной выше неоднородной среде\cite{bib:thomas}. Большой интерес представляет изучение влияния вязкоупругих свойств нефти на различные технологические процессы ее добычи. Если рассмотреть совместное движение различных несмешивающихся жидкостей в трещинах и пористых пластах с учетом вязкоупругих характеристик, то движение вязкоупругой и вязкой жидкостей в плоской горизонтальной трещине без учета поверхностных явлений описывается одномерным гиперболическим уравнением и уравнением теплопроводности с интегро-дифференциальными условиями на границе раздела движущихся жидкостей\cite{bib:gelfand}.

В данной работе, для решения задачи о сопряжении гиперболического и параболического уравнений по пространственной переменной в конечных областях был применен метод конечных элементов. 
На данный момент метод конечных элементов превратился в распространенный способ решения широкого круга научных и инженерных задач. Его развитие стимулировалось как новыми математическими исследованиями, так и проектированием новых плотин, мостов, зданий, воздушных аппаратов, автомобилей, станков и других инженерных объектов.\cite{bib:mke-introduction}

Решение находится с помощью пакета вычислительной математики FEniCS.
Рассматривается погрешность полученного решения и порядок метода конечных элементов


\newpage
