\subsection{Вариационная формулировка задачи сопряжения гиперболо параболического уравнения}

Для применения метода конечных элементов введем следующие пространства:

\begin{equation}
     V = \{u: u_1: x \in \Omega_1; u_2: x \in \Omega_2; u=0, x \in \parom;\} 
     \label{eq:v-space}
\end{equation}

\begin{equation}
     \hat{V} = \{v \in H^1(\Omega), v=0, x \in \parom \} 
     \label{eq:v-hat-space}
\end{equation}


Пространство $H^1$ - пространство Соболева, содержащие такие функции $v$,
что функции $v^2$  и $||\nabla v||^2$ имеют конечные интегралы по области $\Omega$ \\

На нулевом слое нам известно точное значение функции $u$. Это значение можно увидеть из начальных условий (\ref{eq:start1}), (\ref{eq:start3}).

Разберемся с первым слоем. Для $\Omega_1$ нам дано (\ref{eq:start2}). Домножим (\ref{eq:start2}) на $v$ с обеих сторон и проинтегрируем по области $\Omega_1$. Соответственно, получим:
$$
    \int_{\Omega_1}\pdt{u_1}vdx = \int_{\Omega_1}^{}\psi vdx
$$

Распишем первую производную по $t$ по разностной формуле $\pdt{u} = \frac{u^1 - u_0}{\tau}$. Соответственно, получим:
\begin{equation}
    \int_{\Omega_1}^{}\frac{u^1 - u_0}{\tau}vdx = \int_{\Omega_1}^{}\psi vdx
    \label{eq:second-layer}
\end{equation}

Перенесем все части с $u^1$ в (\ref{eq:second-layer}) в левую часть, а остальные - в правую. Получим:
\begin{equation}
    \int_{\Omega_1}^{}u_1^1 vdx= \tau\int_{\Omega_1}^{}\psi vdx + \int_{\Omega_1}^{}\phi_1 vdx
    \label{eq:second-layer-mke-form}
\end{equation}

Из (\ref{eq:second-layer-mke-form}) выделим следующие члены:

\begin{equation}
    a_1(u_1^1,v) = \int_{\Omega_1}^{}u_1^1 vdx
    \label{eq:second-layer-a}
\end{equation}

\begin{equation}
    L_1(v) = \tau\int_{\Omega_1}^{}\psi vdx + \int_{\Omega_1}^{}\phi_1 vdx
    \label{eq:second-layer-l}
\end{equation}

В литературе $a$ называется билинейной формой, а $L$ - линейной формой.

Так как у нас нет условий на $\Omega_2$ для первого слоя, то необходимо использовать (\ref{eq:def2}). Домножим (\ref{eq:def2}) на $v$ с обеих сторон и проинтегрируем по пространству $\Omega_2$. Получим:

\begin{equation}
    \int_{\Omega_2}^{}\frac{u_2^1 - u_2^0}{\tau}vdx = \int_{\Gamma}k_1(\grad u_1^1, \vec{n})vdS - \int_{\Omega_2}k_2(grad u_2^1, gradv)dx + \int_{\Omega_2}f_2vdx
    \label{eq:v-2-mke}
\end{equation}

Для производной $u$ по $t$ использовали разностную подстановку определенную выше. 

Таким образом, для первого слоя для $\Omega_2$ получили следующие линейные и билинейные формы:

\begin{equation}
    a_2(u_2^1, v) = \int_{\Omega_2}^{}\frac{u_2^1 - u_2^0}{\tau}vdx
    \label{eq:second-layer-omega2-a}
\end{equation}

\begin{equation}
    L_2(v) = \int_{Г}^{}k_1(\grad u_1^1, \vec{n})vdS - \int_{\Omega_2}^{}k_2(gradu_2^1, gradv)dx + \int_{\Omega_2}^{}f_2vdx
    \label{eq:second-layer-omega2-L}
\end{equation}


Теперь, когда у нас есть все данные для нулевого и первого слоя - распишем задачу для $n=2,\ldots$.

Вариационная задача строится аналогичным образом: домножаем уравнения (\ref{eq:def1})-(\ref{eq:def2}) на $v \in V$ с обеих сторон
и интегрируем их по пространству. Уравнение (\ref{eq:def1}) интегрируем по пространству
$\Omega_1$, уравнение (\ref{eq:def2}) интегрируем по пространству $\Omega_2$. \\

Для преобразования подыинтегральных выражений вида $ div(\ov a) v$ применяем формулу Остроградского:

$$ div(\ov a) v = div(\ov a v) - (\ov a, gradv)$$

$$ \int_{\Omega_1} \pdtt{u} v dx  = \int_{\Gamma} v(k_1 gradu_1, \ov n) dS - \int_{\Omega_1}(k_1 gradu_1, gradv)dx + \int_{\Omega_1} f_1vdx  $$

Применим начальное условие Дирихле (3) и получим:

\begin{equation}
    \int_{\Omega_1} \pdtt{u} v dx  = \int_\Gamma v(k_1 gradu_1, \ov n) dS - \int_{\Omega_1}(k_1 gradu_1, gradv)dx + \int_{\Omega_1} f_1vdx 
    \label{eq:1-v}
\end{equation}


$\ov n$ - внешняя нормаль к области $\Omega_1$ от границы $\Gamma$.
Члены вида $(grad u, \ov n)$ - производные по направлению от границы. 
Функция $v$ в литературе называется тестовой функцией(Test Function)[3], $u$ - триальной функцией(Trial Function).

Домножим уравнение (2) на тестовую функцию $v$ с обеих сторон, проинтегрируем полученное равенство по области $\Omega_2$ и применим условие, что $v = 0, x \in \parom$ :

\begin{equation}
    \int_{\Omega_2} v\pdt{u_2} = \int_{\parom_2} k_2(x)v(grad u_2, \ov n)dS - \int_{\Omega_2}(k_2(x) gradu_2, grad v) dx + \int_{\Omega_2} f_2vdx
    \label{eq:2-v}
\end{equation}


К уравнению (\ref{eq:1-v}) прибавим (\ref{eq:2-v}), применяя условие сопряжения (\ref{eq:conjugation}):

\begin{equation}
    \int_{\Omega_1} \pdtt{u_1} v dx + \int_{\Omega_2}\pdt{u_2}vdx  = -\int_{\Omega} k(x)(gradu, gradv)dx + \int_{\Omega} fv dx
    \label{eq:generic}
\end{equation}


Таким образом, мы получили уравнение (\ref{eq:generic}) для всей области $\Omega$ вместо
исходных (\ref{eq:def1}), (\ref{eq:def2}) для $\Omega_1$ и $\Omega_2$. Заметим, что (\ref{eq:generic}) равносильно уравнениям (\ref{eq:def1})-(\ref{eq:def2}).\\

Произведем дискретизацию по времени. Разобьем отрезок $[0, T]$ на $N$ частей с шагом $\tau$
и распишем член $\pdtt{u}$ и $\pdt{u}$ по формуле разностной производной:
\begin{equation}
     \pdt{u} = \frac{u^{n+1} - u^{n}}{\tau} 
     \label{eq:pdt-discr}
\end{equation}
\begin{equation}
     \pdtt{u} = \frac{u^{n+1} - 2 u^n + u^{n-1}}{\tau^2} 
     \label{eq:pdtt-discr}
\end{equation}

Домножим на $v$ и проинтегрируем начальные условия (\ref{eq:start2}). Получим:
\begin{equation}
    \int_{\Omega_1}\pdt{u_1}vdx = \int_{\Omega_1}\psi vdx
    \label{eq:psi-v}
\end{equation}

Подставим в (\ref{eq:psi-v}) выражение (\ref{eq:pdt-discr}):
$$\int_{\Omega_1} \frac{u_1^1-u_1^0}{\tau}vdx = \int_{\Omega_1}\psi vdx $$

После нехитрых преобразований получим конструкцию вида:

\begin{equation}
     a(u^{1}, v) = L(v) 
     \label{eq:al-first-layer}
\end{equation}


Где:
$$ a_1(u^{n + 1}) = \int_{\Omega_1}u_1vdx $$
$$ L_2(v) = \int_{\Omega_1}\varphi_1 vdx + \tau\int_{\Omega_1}\psi vdx $$


Таким образом мы получили разложение для $\Omega_1$ на первом слое. Теперь мы должны получить соответствующее разложение для $\Omega_2$. Для этого воспользуемся уравнением (\ref{eq:generic});

$$ \int_{\Omega_2}vdx = \int_{\Gamma}k_1(gradu_1^1, \ov n)vds - \int_{\Omega_2}k_2(gradu_2^1, gradv)dx + \int_{\Omega_2}f_2v_2dx $$

После преобразований получим окончательное выражение линейной и билинейной формы для первого слоя на $\Omega_2$:

$$ a_2(u_2^1, v) = \int_{\Omega_2}vdx + \tau\int_{\Omega_2}^{}k_2(gradu_2^1, gradv)dx $$
$$ L_2(v) = \int_{\Gamma}(gradu_1^1, \ov n)vdS + \int_{\Omega_2}^{}vdx vdx$$

Проделаем аналогичные преобразования(перенос членов в левую и правые части) для (\ref{eq:generic}):

\begin{equation}
    a(u, v) = \int_{\Omega_2}^{}vdx + \tau\int_{\Omega_2}^{}u^{n+1}vdx + \tau^2\int_{\Omega}^{}k(gradu^{n+1}, gradv)dx
    \label{eq:a-variational-generic}
\end{equation}

\begin{equation}
    L(v) = \int_{\Omega_1}^{}(2u^n - u^{n-1})vdx + \tau\int_{\Omega_2}^{}u^nvdx + \tau^2\int_{\Omega}^{}fvdx   
    \label{}
\end{equation}
