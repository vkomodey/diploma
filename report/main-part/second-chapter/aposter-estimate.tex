\subsection{Апостериорная оценка погрешности для решения задачи сопряжения гиперболо параболического уравнения}


Обозначим за $u^n$ - решение задачи (\ref{eq:def1})-(\ref{eq:start3}), а $u_h^n$ - наше решение, полученное методом конечных элементов.

В данном разделе исследована погрешность по методу Рунге-Эйткена[4].
Саму погрешность представим в следующем разложении:
\begin{equation}
    E_0 = Mh^p
    \label{eq:error}
\end{equation}
В (\ref{eq:error}) коэффициент $M$ - некая константа, которая определяется методом решения. $h$ - соответственно, шаг дифференцирования. $p$ - порядок метода.
В свою очередь - $E_0$ называется главным членом погрешности. 

Распишем следующую величину:
\begin{equation}
     ||e_h|| = ||u_h^n - u^n|| = ||u_h^n|| + M_1h^p + O(h^{p+1}) 
    \label{eq:e_h}
\end{equation}

Теперь, вычислим ту же самую разность, но уже с новым шагом $kh$:

\begin{equation}
     ||e_{kh}|| = ||u_{kh} - u^n|| = ||u_{kh}|| + M(kh)^p + O((kh)^{p+1}) 
    \label{eq:e_kh}
\end{equation}

Где коэффициент пропорциональности $k$ может быть как больше, так и меньше единицы. Коэффициент $M$ будет одинаковым, так как вычисляется одна и та же переменная, одним и тем же методом, а от величины шага $M$ не зависит.

Пренебрегая бесконечно малыми величинами, приравняем (\ref{eq:e_h}) и (\ref{eq:e_kh}):
$$
||u_h^n|| + E_0 = ||u_{kh}|| + k^pE_0
$$

Откуда найдем главный член погрешности:

\begin{equation}
    E_0 = \frac{||u_h^n|| - ||u_{kh}||}{k^p - 1}
    \label{eq:runge}
\end{equation}

Формула (\ref{eq:runge}) называется первой формулой Рунге и она позволяет оценить погрешность. Формула (\ref{eq:runge}) имеет большое практическое значение, так как позволяет провести оценку погрешности без изменения алгоритма метода.

В нашем случае неизвестен порядок метода -- степень $p$. Для этого необходимо третий раз вычислить значение величины $e_h$ с шагом $k^2h$, то есть:
\begin{equation}
     ||e_h|| = ||u_{k^2h}|| + k^{2p}E_0
    \label{eq:e_h-norma}
\end{equation}

Приравния правые части (\ref{eq:runge}) и (\ref{eq:e_h-norma}), можем выразить $k^p$:
\begin{equation}
    k^p = \frac{||u_{kh}|| - ||u_{k^2h}||}{||u_{h}|| - ||u_{kh}||}
    \label{eq:kp}
\end{equation} 
\soeq{}

Прологарфмируя соотношение (\ref{eq:kp}) определим порядок $p$.

\begin{equation}
    p =\frac{ ln \left(\frac{||u_{kh}|| - ||u_{k^2h}||}{||u_{h}|| - ||u_{kh}||}\right)}{ln k}
    \label{eq:mke-level}
\end{equation}
