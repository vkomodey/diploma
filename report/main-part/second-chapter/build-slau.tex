\subsection{Построение СЛАУ для задачи сопряжения гиперболо параболического уравнения}

Для решения нашей задачи численно, необходимо применить процесс дискретизации по пространству 
к исходной задаче (\ref{eq:def1})-(\ref{eq:start3}). Функцию, которую будем находить обозначим $u_h$.

Далее, область $\Omega$ мы разобьем на конечные элементы. 
В нашем случае мы будем разбивать область на треугольники. Процесс называется триангуляцией(в данной работе использовалась триангуляция Делоне).
Разобьем область на $M$ конечных элементов и зададим на каждом базисную функцию $\phi_k$. Причем:
\begin{equation}
    u_h(x) = \sum_{k=1}^Mu_k^{n+1}\phi_k(x)
    \label{eq:u-interpolation}
\end{equation}

Где:

\begin{equation}
    v(x) = \phi_j
    \label{eq:v-phi}
\end{equation}

Как известно из метода конечных элементов, мы оперируем линейной и билинейной формой. 
\begin{equation}
    a(u, v) = L(v)
    \label{eq:a-v-generic}
\end{equation}

Подставим в (\ref{eq:a-v-generic}) выражения (\ref{eq:u-interpolation}) и (\ref{eq:v-phi}) соответственно. Получим следующее равенство:

\begin{equation}
    a(u^{n+1}, \phi_i) = a(\sum_{k=1}^{M}u_k^{n+1}\phi_k, \phi_i)
    \label{eq:a-interpolation}
\end{equation}

Известно, что оператор $a$ линеен, то есть:
$$ a(\lambda_1u_1 + \lambda_2u_2, v) = \lambda_1a(u_1, v) + \lambda_2a(u_2, v) $$

Таким образом, перепишем (\ref{eq:a-interpolation}):
\begin{equation}
    a(u^{n+1}, \phi_i) = \sum_{k=1}^{M}u_k^{n+1}a(\phi_k, \phi_i)
    \label{a-slau1}
\end{equation}

Таким образом, получили следующую задачу: $AU=F$, где матрица A состоит из членов $a_{ki} = a(\phi_k, \phi_i)$.
Искомый вектор U имеет вид:

$$ U = \begin{bmatrix} u_0^{n+1} \\ u_1^{n+1} \\ \cdots \\ u_M^{n+1} \end{bmatrix} $$

Получили систему, которую, например, можно решить методом LU факторизации.
