\subsection{Постановка задачи сопряжения для уравнения гиперболо параболического типа}
Пусть ограниченная область $\Omega \subset  \mathbb{R}^2 $ с границей $\partial \Omega$ разбивается кривой 
$\Gamma$ на две подобласти $\Omega_1$ и $\Omega_2$. В области $\Omega_2$ будем рассматривать уравнение параболического типа,
 а в $\Omega_1$ -- уравнение гиперболического типа по $t$. На $\partial \Omega$ задаются граничные условия, на $\Gamma$ -- условия сопряжения. Задача формулируется следующим образом: для неизвестной функции $u$:
$$
u(x) =
  \begin{cases}
    u_1, \quad x \in \Omega_1\\
    u_2, \quad x \in \Omega_2 \\
  \end{cases}
$$
рассмотрим следующие уравнения:
\begin{equation}
    \pdtt{u_1} = div(k_1(x)gradu_1) + f_1(x, t), \quad x \in \Omega_1,~t > 0,
    \label{eq:def1}
\end{equation}

\begin{equation}
    \pdt{u_2} = div(k_2gradu_2) + f_2(x, t), \quad x \in \Omega_2,~t>0,
    \label{eq:def2}
\end{equation}

В (\ref{eq:def1}) - (\ref{eq:def2}) коэффициент $k(x)$ равен:
$$
k(x) =
  \begin{cases}
    k_1, \quad x \in \Omega_1\\
    k_2, \quad x \in \Omega_2 \\
  \end{cases}
$$

Уравнения (1) и (2) дополним граничными условиями Дирихле:
\begin{equation}
     u = 0, \quad x \in \parom.
    \label{eq:dirichlet}
\end{equation}

В (\ref{eq:dirichlet}) $\parom$ - граница области $\Omega$. \\

На границе разделов, в области сопряжения задаются следующие условия:
\begin{equation}
    (k_1(x)gradu_1, \ov n) = (k_2(x)gradu_2, \ov n), \quad x \in \Gamma ,
    \label{eq:conjugation}
\end{equation}
$$ u_1(x,t) = u_2(x, t), \quad x \in \Gamma$$

Также в области $\Omega_1$ и $\Omega_2$ задаются начальные условия:
\begin{equation}
     u_1(x, 0) = \varphi_1(x), \quad x \in \Omega_1 
     \label{eq:start1}
\end{equation}

\begin{equation}
     \pdt{u_1}(x, 0) = \psi(x), \quad x \in \Omega_1 
     \label{eq:start2}
\end{equation}

\begin{equation}
     u_2(x, 0) = \varphi_2(x), \quad x \in \Omega_2
     \label{eq:start3}
\end{equation}


Известно, что задача (\ref{eq:def1}) -  (\ref{eq:start3}) имеет единственное решение 
